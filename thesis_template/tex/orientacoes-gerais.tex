

\section{Codificação dos arquivos: UTF8}

A codificação de todos os arquivos do \abnTeX\ é \texttt{UTF8}. É necessário que
você utilize a mesma codificação nos documentos que escrever, inclusive nos
arquivos de base bibliográficas |.bib|.



\section{Inclusão de outros arquivos}\label{sec-include}

É uma boa prática dividir o seu documento em diversos arquivos, e não
apenas escrever tudo em um único. Esse recurso foi utilizado neste
documento. Para incluir diferentes arquivos em um arquivo principal,
de modo que cada arquivo incluído fique em uma página diferente, utilize o
comando:

\begin{verbatim}
   \include{documento-a-ser-incluido}      % sem a extensão .tex
\end{verbatim}

Para incluir documentos sem quebra de páginas, utilize:

\begin{verbatim}
   \input{documento-a-ser-incluido}      % sem a extensão .tex
\end{verbatim}



%\section{Remissões internas}

%Ao nomear a \autoref{tab-nivinv} e a \autoref{fig_circulo}, apresentamos um exemplo de remissão interna, que também pode ser feita quando indicamos o \autoref{cap_exemplos}, que tem o nome \emph{\nameref{cap_exemplos}}. O número do capítulo indicado é \ref{cap_exemplos}, que se inicia à \autopageref{cap_exemplos}\footnote{O número da página de uma remissão pode ser obtida também assim: \pageref{cap_exemplos}.}.
%Veja a \autoref{sec-divisoes} para outros exemplos de remissões internas entre seções, subseções e subsubseções.

%O código usado para produzir o texto desta seção é:

%\begin{verbatim}
%Ao nomear a \autoref{tab-nivinv} e a \autoref{fig_circulo}, apresentamos um
%exemplo de remissão interna, que também pode ser feita quando indicamos o
%\autoref{cap_exemplos}, que tem o nome \emph{\nameref{cap_exemplos}}. 
%O número
%do capítulo indicado é \ref{cap_exemplos}, que se inicia à
%\autopageref{cap_exemplos}\footnote{O número da página de uma remissão pode ser
%obtida também assim:
%\pageref{cap_exemplos}.}.
%Veja a \autoref{sec-divisoes} para outros exemplos de remissões internas entre
%seções, subseções e subsubseções.
%\end{verbatim}



\section{Consulte o manual da classe \textsf{abntex2}}

Consulte o manual da classe \textsf{abntex2} \cite{abntex2classe} para uma
referência completa das macros e ambientes disponíveis. 

Além disso, o manual possui informações adicionais sobre as normas ABNT
observadas pelo \abnTeX\ e considerações sobre eventuais requisitos específicos
não atendidos, como o caso da \citeonline[seção 5.2.2]{NBR14724:2011}, que
especifica o espaçamento entre os capítulos e o início do texto, regra
propositalmente não atendida pelo presente modelo.



\section{Precisa de ajuda?}

Consulte a FAQ com perguntas frequentes e comuns no portal do \abnTeX:
\url{https://code.google.com/p/abntex2/wiki/FAQ}.

Inscreva-se no grupo de usuários \LaTeX:
\url{http://groups.google.com/group/latex-br}, tire suas dúvidas e ajude
outros usuários.

Participe também do grupo de desenvolvedores do \abnTeX:
\url{http://groups.google.com/group/abntex2} e faça sua contribuição à
ferramenta.



\section{Você pode ajudar?}

Sua contribuição é muito importante! Você pode ajudar na divulgação, no
desenvolvimento e de várias outras formas. Veja como contribuir com o \abnTeX\
em \url{https://code.google.com/p/abntex2/wiki/ComoContribuir}.